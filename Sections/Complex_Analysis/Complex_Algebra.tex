\subsection{Complex Algebra}
Complex numbers arise from the roots of polynomials.\\
Ex: $x^2+1=0\Ra x^2=-1 \Ra x=\pm\sqrt{-1}$. This polynomial has no real roots, however, we can introduce an imaginary number $i$ such that $i^2=-1$. Then we will have the solution $x=\pm i$\\
We can introduce \textit{complex numbers} which are numbers in the form $z=x+iy$, where $x$ is the real part of $z$, $\Re(z)$, and $y$ is the imaginary part of $z$, $\Im(z)$. 
These numbers can also be expressed in vector notation along the complex plane.


\subsubsection{Complex Arithmetic}
Addition, subtraction, and multiplication work the same, just with the addition of the fact $i^2=-1$. For division, we require what is called the conjugate.\\
The conjugate of a complex number is the same number, just with the sign of the imaginary component flipped.
$$\overline{z}=x-yi$$
where $\overline{z}$ is the conjugate of $z$.\\
Similarly to vectors, we can also define the modulus (length) of a complex number
$$|z|^2=x^2+y^2=z\cdot\overline{z}$$
Using this, we can define the division of a complex number and also define the real and imaginary components of a complex number.\\
The general expression for division is:
\begin{align*}
    &\frac{u}{z}=\frac{s+it}{x+iy}=\frac{(s+it)(x-iy)}{(x+iy)(x-iy)}=\frac{u\overline{z}}{x^2+y^2}=\frac{u\overline{z}}{|z|^2}
\end{align*}
The real and imaginary components can be computed as
\begin{align*}
    &z=x+iy\\
    &\overline{z}=x-iy\\
    &\Re(z)=\frac{z+\overline{z}}{2}\\
    &\Im(z)=\frac{z-\overline{z}}{2i}
\end{align*}
Ex1: Simplify $(1+2i)(3+i)(2-3i)$
\begin{align*}
    &(1+2i)(3+i)(2-3i)=(3+i+6i-2)(2-3i)=(1+7i)(2-3i)\\
    &=2-3i+14i+21\\
    &=23+11i
\end{align*}
Ex2: Simplify $\bfrac{2+i}{1+i}^2$
\begin{align*}
    &\bfrac{2+i}{1+i}^2=\brround{\frac{(2+i)(1-i)}{(1+i)(1-i)}}^2=\bfrac{2+i-2i+1}{2}^2=\bfrac{3-i}{2}^2=\frac{9-6i-1}{4}\\
    &=2-\frac{3}{2}i
\end{align*}
Ex3: Simplify $(1+2i)^5$
\begin{align*}
    &(1+2i)^5=1^5+5(1)^4(2i)+10(1)^3(2i)^2+10(1)^2(2i)^3+5(1)(2i)^4+(2i)^5\\
    &=1+10i+10(-4)+10(-8i)+5(16)+32i\\
    &=1+10i-40-80i+80+32i\\
    &=41-38i
\end{align*}
Ex4: Prove
\[ \text{if }|z|=1\text{ then }\Re\bfrac{1}{1+z}=\frac{1}{2} \]
\begin{proof}
\begin{align*}
    &\Re(w)=\frac{w+\overline{w}}{2}\\
    &\overline{\frac{1}{1+z}}=\frac{1}{1+\overline{z}}\\
    &\Re\bfrac{1}{1+z}=\frac{\frac{1}{1+z}+\frac{1}{1+\overline{z}}}{2}=\frac{1+\overline{z}+1+z}{2(1+z)(1+\overline{z})}=\frac{2+z+\overline{z}}{2(1+z+\overline{z}+|z|^2)}\\
    &|z|=1\Ra \Re\bfrac{1}{1+z}=\frac{2+z+\overline{z}}{2(2+z+\overline{z})}=\frac{1}{2}
\end{align*}
\end{proof}
Some properties of $\overline{z}$:
\begin{itemize}
    \item $\overline{\overline{z}}=z$
    \item $\overline{z_1z_2}=\overline{z}_1\overline{z}_2$
    \item $|z_1z_2|=|z_1||z_2|$
\end{itemize}
Some common inequalities come from the triangle inequality:\\
$|z_1+z_2|\leq |z_1|+|z_2|$
\begin{proof}
\begin{align*}
    &|z_1+z_2|\leq |z_1|+|z_2|\\
    &|z_1+z_2|=\sqrt{(x_1+x_2)^2+(y_1+y_2)^2}\\
    &\sqrt{(x_1+x_2)^2+(y_1+y_2)^2}\leq \sqrt{x_1^2+y_1^2}+\sqrt{x_2^2+y_2^2}\\
    &|z_1+z_2|\leq |z_1|+|z_2|
\end{align*}
\end{proof}
\[
\Ra |z_1\pm z_2|\geq ||z_1|-|z_2||
\]
This leads to a general upper and lower bound that can be derived from these inequalities:
$$||z_1|-|z_2||\leq|z_1\pm z_2|\leq |z_1|+|z_2|$$


Another way to represent complex numbers is through polar form. To use this we must first introduce Euler's identity:
$$e^{i\varphi}=\cos\varphi+i\sin\varphi$$
This then helps us with the equation for the polar form
$$z=a+ib=|z|(\cos\varphi+i\sin\varphi)=|z|e^{i\varphi}$$
For $z=re^{i\varphi}$ We call $r$ the magnitude of the complex number and $\varphi$ is the argument. This can be especially useful for simplifying some complex numbers.\\
Ex:
\begin{align*}
    &\brvertical{\frac{(1+\sqrt{3}i)^{100}}{(\sqrt{3}-i)^{100}}}=\frac{|1+\sqrt{3}i|^{100}}{|\sqrt{3}-i|^{100}}=\frac{2^{100}}{2^{100}}=1
\end{align*}
Note that because sinusoidal functions are periodic every $2\pi$ this means that there infinite ways to express a function in polar coordinates. 
\[
e^{i2\pi k}=1,\ k\in\Z\Ra z=re^{i(\varphi+2\pi k)}
\]
To get around the issue of having infinite possible polar forms for every complex number we define what's called the principal argument to be
$$\Arg(z)=\varphi\in(-\pi,\pi]$$
We define the regular argument to be
$$\arg(z)=\Arg(z)+2k\pi,\ k\in\Z$$
Note that $\Arg(z)$ is singular valued while $\arg(z)$ is multi-valued.\\
We can define $\Arg(z)$ in terms of the real and imaginary parts ($x$ and $y$) as
$$\Arg(z)=\arctan(\tfrac{y}{x})\pm k\pi$$
What it is specifically depends on what quadrant of the complex plane the point lies in.
\begin{itemize}
    \item QI: $\Arg(z)=\arctan(\tfrac{y}{x})$
    \item QII: $\Arg(z)=\arctan(\tfrac{y}{x})+\pi$
    \item QIII: $\Arg(z)=\arctan(\tfrac{y}{x})-\pi$
    \item QIV: $\Arg(z)=\arctan(\tfrac{y}{x})$
\end{itemize}
Ex:
\begin{align*}
    &\Arg(-1-\sqrt{3}i)=\arctan(\sqrt{3})-\pi=-\frac{2\pi}{3}
\end{align*}
Ex2:
\begin{align*}
    &\arg(1-\sqrt{3}i)\\
    &\Arg(1-\sqrt{3}i)=-\frac{\pi}{3}\\
    &\arg(1-\sqrt{3})=-\frac{\pi}{3}+2k\pi,\ k\in\Z
\end{align*}
Ex3:
\begin{align*}
    &\arg(-1+2i)\\
    &\Arg(-1+2i)=\arctan(-2)+\pi\\
    &\arg(-1+2i)=\pi-\arctan(2)+2k\pi
\end{align*}
Ex4: Simplify
\begin{align*}
    &z=-3+3i\\
    &\Arg(z)=\arctan\bfrac{3}{-3}+\pi=\frac{3\pi}{4}\\
    &|z|=3\sqrt{2}\\
    &z=3\sqrt{2}e^{\frac{3\pi}{4}i}
\end{align*}
Ex5: Simplify
\begin{align*}
    &z=-3-3i\\
    &\Arg(z)=\arctan\bfrac{-3}{-3}-\pi=-\frac{3\pi}{4}\\
    &|z|=3\sqrt{2}\\
    &z=3\sqrt{2}e^{-\frac{3\pi}{4}i}
\end{align*}
Ex6: Simplify
\begin{align*}
    &z=\frac{1-i}{-\sqrt{3}+i}=\frac{u}{v}\\
    &\arg(z)=\arg(u)-\arg(v)=\arctan\bfrac{-1}{1}-\brround{\arctan\bfrac{1}{-\sqrt{3}}+\pi}+2k\pi\\
    &=-\frac{\pi}{4}-\frac{5\pi}{6}+2k\pi=-\frac{13\pi}{12}+2k\pi\\
    &\Arg(z)=\frac{11\pi}{12}\\
    &|z|=\frac{|u|}{|v|}=\frac{\sqrt{2}}{2}=\frac{1}{\sqrt{2}}\\
    &z=\frac{e^{\frac{11\pi}{12}i}}{\sqrt{2}}
\end{align*}
Ex7: Simplify
\begin{align*}
    &z=(\sqrt{3}-i)^2=w^2\\
    &\arg(w)=\arctan\bfrac{-1}{\sqrt{3}}+2k\pi=-\frac{\pi}{6}\\
    &\Ra \arg(z)=2\arg(w)=-\frac{\pi}{3}+4k\pi\\
    &\Arg(z)=-\frac{\pi}{3}\\
    &|w|=2\Ra |z|=|w|^2=4\\
    &z=4e^{-\frac{\pi}{3}i}
\end{align*}

Properties of $\Arg(z)$ and $\arg(z)$
\begin{itemize}
    \item $\Arg(z_1z_2)\neq\Arg(z_1)+\Arg(z_2)$
\begin{proof}
Proof by contradiction: assume that $\Arg(z_1z_2)=\Arg(z_1)+\Arg(z_2)$ is true.\\
Take $z_1=z_2=-1$.
\begin{align*}
    &\Arg(z_1)=\Arg(z_2)=\pi\\
    &\Ra \Arg(z_1)+\Arg(z_2)=2\pi\\
    &\Arg(z_1z_2)=\Arg(1)=0\\
    &\Arg(z_1z_2)\neq\Arg(z_1)+\Arg(z_2)\ \forall z_1,z_2\neq0\in\C
\end{align*}
\end{proof}
\item $\arg(z_1z_2)=\arg(z_1)+\arg(z_2)$
\item $ \Arg(\overline{z})\neq-\Arg(z) $
\begin{proof}
Proof by contradiction: assume that $\Arg(\overline{z})=-\Arg(z)$ is true.\\
Take $z=-1$
\begin{align*}
    &\overline{z}=z=-1\\
    &\Arg(z)=\pi\\
    &\Arg(\overline{z})=\pi\\
    &-\Arg(z)=-\pi\\
    &\Ra \Arg(\overline{z})\neq-\Arg(z)\ \forall z\in\C
\end{align*}
\end{proof}
\item $ \arg(z)=-\arg(\overline{z}) $
\begin{proof}
\begin{align*}
    &z=|z|e^{i\arg(z)}\ \forall z\in\C\\
    &\overline{z}=|z|e^{-i\arg(z)}\\
    &\Ra \arg(\overline{z})=-\arg(z)
\end{align*}
\end{proof}
\end{itemize}
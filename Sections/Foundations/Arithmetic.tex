\subsection{Arithmetic}

\subsubsection{Addition}
For an addition statement, $a+b=c$,\\
$a$ and $b$ are called addends.\\
The result, $c$ is called the sum.
\begin{itemize}
    \item Commutative property of addition:\\
    changing the order of addends does not change the sum.
    $$a+b=b+a$$
    \item Associative property of addition:\\
    changing the grouping of addends does not change the sum.
    $$a+(b+c)=(a+b)+c$$
\end{itemize}



\subsubsection{Subtraction}
Subtraction is the opposite operation of addition.\\
For a subtraction statement, $a-b=c$,\\
$a$ is called the minuend.\\
$b$ is called the subtrahend.\\
The result, $c$, is called the difference.

\subsubsection{Multiplication}
Multiplication is analogous to repeated addition.\\
Ex: $a+a+a=3a$\\
For a multiplication statement, $ab=c$,\\
$a$ and $b$ are called factors.\\
The result, $c$, is called the product.\\
\textbf{Commutative property of multiplication:} changing the order of factors does not change the product.
$$ab=ba$$
\textbf{Associative property of multiplication:} changing the grouping of factors does not change the product.
$$a(bc)=(ab)c$$
Multiplying by 1: any number multiplied by one will just be that number.
$$1a=a$$
Multiplying by 0: any real number multiplied by zero will become zero.
$$0a=0$$

Long Multiplication:\\
Ex: $21\cdot 42$\\
-Can break it up into 4 multiplication statements and sum the products
\begin{flalign*}
    & 20\cdot 40 = 800 & \\
    & 20\cdot 2 = 40 & \\
    & 1\cdot 40 = 40 & \\
    & 1\cdot 2 = 2 &
\end{flalign*}

\subsubsection{Division}
Division is the opposite operation of multiplication.\\
For a division statement, $a\div b=c$ or $\frac{a}{b}=c$\\
$a$ is called the numerator or the dividend.\\
$b$ is called the denominator or the divisor\\
The result, $c$ is called the quotient.\\
Any amount leftover that doesn't divide evenly is called the remainder.\\
Dividing by 1: any number divided by 1 is itself.
$$a\div 1=a$$
Zero divided by any real number is zero.
$$0\div a=0$$

Long Division:\\
Ex: $412\div 5$\\
\begin{align*}
  \renewcommand\arraystretch{.75}\renewcommand\arraycolsep{3pt}
  \begin{array}{r@{\hskip\arraycolsep}c@{\hskip\arraycolsep}l*5r} % n=8=3+5
    &&&8&2\\
    \cline{2-5} %n=8
   5&\Bigg)&4&1&2\\
    &&4&0&&\\
    \cline{3-5}\\
    &&&1&2&\\
    &&&1&0&\\
    \cline{4-5}\\
    &&&&2\\
  \end{array}
\end{align*}
Remainder is 2\\
So the answer is $412\div 5=82+\frac{2}{5}$

\subsubsection{Factors}
A factor is a whole number that can divide evenly into another number.\\
Factor pairs are two numbers that multiply to a certain product.\\
Ex: factor pairs of 8 are \{1,8\} and \{2,4\}\\
Even numbers are those that are evenly divisible by 2 (e.g. 2, 4, 6,8, 10). Odd numbers are numbers that are not even;y divisible by 2. (e.g. 1, 3, 5, 7, 9).\\
Prime numbers are whole numbers greater than 1 that cannot be exactly divided by any whole number other than itself and 1 (e.g. 2, 3, 5, 7, 11).\\
\\
Prime Factorization:\\
Prime factorization is writing a number as a product of all its prime number factors.\\
Ex: $36=3\cdot 3\cdot 3\cdot 3\cdot 2\cdot 2$\\
\\
Least Common Multiple (LCM):\\
It is the smallest whole number that all numbers in a given set can divide evenly into.\\
Ex: What is the LCM of 12 and 18?\\
$12=3\cdot 2\cdot 2$\\
$18=3\cdot 3\cdot 2$\\
The prime factors of the LCM must also contain the prime factors of each number in the set.\\
$\therefore $ the prime factorization is $3\cdot 3\cdot 2\cdot 2$\\
$\therefore$ the LCM is 36.\\
\\
Greatest Common Factor (GCF):\\
It is the largest number that all numbers in a given set can be divided by.\\
Ex: What is the GCF of 12 and 18?\\
$12=3\cdot 2\cdot 2$\\
$18=3\cdot 3\cdot 2$\\
The GCF can be found by taking the product of the prime factors that all numbers in a set have in common.\\
12 and 18 have factors 2 and 3 in common,\\
$\therefore$ the GCF is 6.\\
\\
Factoring:
Factoring is when you take a number or an expression and break it up into factors. This is usually done for the purpose of taking out a GCF in order to simplify, or it is done as a step in solving equations.\\
Ex: $14+6=2(7+3)$
\subsection{Fractional Numbers}

\subsubsection{Fraction Notation}
Fractions are an unsolved division statement expressed as $\frac{a}{b}$.\\
We usually like to express fractions in their simplest form. To do this, we factor out the GCF from both the numerator and denominator and cancel the GCF.\\
Ex: $\frac{8}{12}=\frac{4(2)}{4(3)}=\frac{2}{3}$ where $\frac{2}{3}$ is in simplest form.\\
\\
When a number has a value greater than 1, it can be expressed as either a mixed number or as an improper fraction.\\
Mixed numbers are a whole number plus the remaining fraction. Ex: $2\frac{1}{2}$\\
Improper fractions are where the numerator is larger than the denominator. Ex: $\frac{5}{2}$\\
Note that the value of the two above examples are equivalent.\\

\subsubsection{Adding and Subtracting Fractions}
Step 1: The denominator of both fractions must be the same. If they are not the same then you will need to find the LCM which will become the new denominator for both fractions.\\
Step 2: Add or subtract the numerators accordingly, leaving the denominator the same.\\
Ex: $\frac{3}{12}+\frac{5}{18}$\\
The LCM is 36 so we can rewrite\\
$\frac{3}{12}\left(\frac{3}{3}\right)+\frac{5}{18}\left(\frac{2}{2}\right)=\frac{9}{36}+\frac{10}{36}=\frac{19}{36}$\\

\subsubsection{Multiplying and Dividing Fractions}
Multiplying fractions is easy. You can simply multiply the numerators and multiply the denominators.\\
Ex: $\frac{1}{3}\cdot\frac{2}{3}=\frac{2}{18}=\frac{1}{9}$\\
\\
A reciprocal is defined to be where the numerator and denominator are flipped. So the reciprocal of $\frac{a}{b}$ would be $\frac{b}{a}$.\\
When dividing fractions, we take the reciprocal of the divisor and then treat it as a multiplication statement.\\
Ex: $\frac{1}{6}\div\frac{2}{3}=\frac{1}{6}\cdot\frac{3}{2}=\frac{3}{12}=\frac{1}{4}$

\subsubsection{Decimals}
Decimals are a place value extension for number of place values smaller than one.\\
The name of each place value (tenths, hundredths, etc.) refers to what that number must be divided by in order to be expressed as a fraction.\\
For example, 0.25 has a place value of hundredths, $\therefore 0.25=\frac{25}{100}=\frac{1}{4}$.\\
\\
A percent is a decimal that is expressed as a fraction over 100. It is meant to represent a part of a whole.\\
Ex: $25\%=0.25\cdot 100\%=0.25=\frac{1}{4}$\\

\subsubsection{Ratios and Rates}
A ratio is a quantity, $a$, of one thing compared to a quantity, $b$ of another. Denoted by $a:b$.\\
A rate is a ratio that compares two quantities of different units of measure, expressed as the number of units of the first quantity for every one unit of the second quantity. A common example of this is velocity which is measured in meters per second.\\
\\
Unit conversions:\\
Unit conversions are used to transfer a measure of one type of unit to a different unit that measures the same thing.\\
Ex: Express $2\units{m/s}$ in $\units{km/h}$.\\
$\frac{2\units{m}}{\units{s}}\cdot\frac{60\units{s}}{1\units{min}}\cdot\frac{60\units{min}}{1\units{h}}\cdot\frac{1\units{km}}{1000\units{m}}=7.2\units{km/h}$
\subsection{Lines and Angles}

\subsubsection{Types of Lines}
A line extends forever in both directions.\\
A segment is a part of a line but with a defined starting and stopping point.\\
A ray is a line with a defined starting point but no defined ending point.\\
Parallel lines are always the same distance apart from one another. No matter how far they extend, they will never meet.\\
Perpendicular lines are lines that meet at right angles.

\subsubsection{Types of Angles}
An angle is formed from intersecting lines. The wider an angle is, the greater its measure.\\
Types of angles:\\
\begin{tabular}{rl}
Acute: & $0^\circ <\theta < 90^\circ$\\
Right: & $\theta = 90^\circ$\\
Obtuse: & $90^\circ < \theta < 180^\circ$\\
Straight: & $\theta = 180^\circ$\\
Reflex: & $180^\circ < \theta < 360^\circ$\\
Circle: & $\theta=360^\circ$
\end{tabular}\\
Complimentary angles are two angles with a sum of $90^\circ$.\\
$\alpha+\beta=90^\circ$\\
Supplementary angles are two angles that sum to $180^\circ$.\\
$\alpha+\beta=180^\circ$

\subsubsection{Units of Angles}
Other units of angles include radians and gradians where radians are the natural unit of measure and gradians are measured with respect to percentages.\\
The conversion factor is:
$$180^\circ=\pi=200\%$$
Common conversions:\\
\begin{tabular}{c|c|c}
    Degrees & Radians & Gradians\\
    \hline
    $30^\circ$ & $\pi/6$ & $33.\overline{3}\%$\\
    $45^\circ$ & $\pi/4$ & $50\%$\\
    $60^\circ$ & $\pi/3$ & $66.\overline{6}\%$\\
    $90^\circ$ & $\pi/2$ & $100\%$\\
    $180^\circ$ & $\pi$ & $200\%$\\
    $270^\circ$ & $3\pi/2$ & $300\%$\\
    $360^\circ$ & $2\pi$ & $400\%$
\end{tabular}
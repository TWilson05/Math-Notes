\subsection{Powers and Order of Operations}

\subsubsection{Exponents}
An exponent is a form of repeated multiplication, expressed in the form $a^b$. It tells us to multiply the base, $a$, by itself $b$ times.\\
Ex: $4^3=4\cdot 4\cdot 4=64$\\
\\
1 to any exponent will be 1.\\
$1^a=1$\\
Any number to exponent 1 will remain the same.\\
$a^1=a$\\
Any number to exponent 0 will be 1.\\
$a^0=1$

\subsubsection{Radicals}
Radicals are the opposite operation of exponents, in the form $\sqrt[b]{a}$. It gives us a value that, when multiplied by itself as many times as specified by the radical, $b$, gives us the base, $a$.\\
Ex: $\sqrt{9}=3$\\
Ex2: $\sqrt[3]{8}=2$\\
Note that the root of 1 will always be 1.\\
$\sqrt[b]{1}=1$

\subsubsection{Order of Operations}
This is the order in which to apply different mathematical operations within calculations. The order is as follows:
\begin{enumerate}
    \item Brackets and grouping symbols
    \item Exponents and radicals
    \item Multiplication and division
    \item Addition and subtraction
\end{enumerate}

\subsubsection{Real Number System}
Real numbers, $\mathbb{R}$, are all numbers that occur naturally and that we can visualise. They are split into the following sub-categories:\\
Natural Numbers, $\mathbb{N}$: counting numbers; numbers that do not need to be represented as a fraction or decimal and are larger than zero.\\
Whole Numbers, $\mathbb{N}_\mathrm{0}$: Whole numbers as well as zero.\\
Integers, $\mathbb{Z}$: All whole numbers and their opposites (negatives).\\
Rational Numbers, $\mathbb{Q}$: Any number that can be expressed as a fraction. (Includes repeating or terminating decimals).\\
Irrational Numbers, $\mathbb{A}_\mathrm{R}$: All remaining real numbers that cannot be expressed as a fraction.
\subsection{Power Laws}

\subsubsection{Exponent Laws}
Product Rule: when multiplying two like variables, you add the exponents.\\ Ex: $x^2\cdot x=x^3$
$$a^x\cdot a^y=a^{x+y}$$
Quotient Rule: when dividing two like variables, you subtract the exponents.\\
Ex: $\frac{x^2}{x}=x$
$$\frac{a^x}{a^y}=a^{x-y}$$
Power Rule: when you have an exponent to a power, the two exponents multiply.\\
Ex: $(x^3)^2=x^6$
$$(a^x)^y=a^{xy}$$
Power of a Product Rule: the exponent of a product is applied to all factors.\\
Ex: $(xy)^2=x^2y^2$
$$(ab)^x=a^xb^x$$
Power of a Fraction Rule: same as the power of a product rule - it also applies to division.\\
Ex: $\left(\frac{x}{3}\right)^2=\frac{x^2}{9}$
$$\left(\frac{a}{b}\right)^x=\frac{a^x}{b^x}$$
Negative Exponents: if an exponent is negative, it takes the reciprocal of the number.\\
Ex: $x^{-2}=\frac{1}{x^2}$
$$a^{-x}=\frac{1}{a^x}$$
Fractional Exponent: If an exponent is in the form of a fraction, the numerator refers to the exponent of a number and the denominator refers to the root.\\
Ex: $x^{2/3}=(\sqrt[3]{x})^2$
$$a^{\frac{x}{y}}=\left(\sqrt[y]{a}\right)^x$$

\subsubsection{Simplifying Radicals}
As we have seen, radicals are a special type of exponent so we can apply the same rules.\\
Product of Radicals:
$$\sqrt[x]{a}\sqrt[x]{b}=\sqrt[x]{ab}$$
Quotient of Radicals:
$$\frac{\sqrt[x]{a}}{\sqrt[x]{b}}=\sqrt[x]{\frac{a}{b}}$$
Radical of a Radical:
$$\sqrt[x]{\sqrt[y]{a}}=\sqrt[xy]{a}$$
Using these rules, we can simplify certain radicals.\\
Ex: $\sqrt{20}$\\
We can break the radical up into its largest perfect square (or cube, or whatever the radical specifies).\\
$\sqrt{20}=\sqrt{4}\sqrt{5}$\\
Then we can take the definitive value of that perfect square, removing the radical.\\
So $2\sqrt{5}$ will be the simplified root.

\subsubsection{Rationalizing}
Sometimes we want to change the appearance of a fraction with a radical to make it easier to work with. We can do this by multiplying by 1 (or $\frac{a}{a}$).\\
Ex: $\frac{1}{\sqrt{3}}=\frac{1}{\sqrt{3}}\left(\frac{\sqrt{3}}{\sqrt{3}}\right)=\frac{\sqrt{3}}{3}$\\
This becomes more complicated if we are dealing with a binomial. In this case we will need to multiply by the conjugate of the denominator. The conjugate of a binomial is where the second term of the binomial switches signs.\\
Ex: conjugate of $x+3$ is $x-3$.\\
Now let's use the conjugate to simplify a binomial fraction.\\
Ex: $\frac{1}{x+\sqrt{2}}=\frac{1}{x+\sqrt{2}}\left(\frac{x-\sqrt{2}}{x-\sqrt{2}}\right)=\frac{x-\sqrt{2}}{x^2-2}$

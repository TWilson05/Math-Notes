\subsection{Arithmetic with Variables}

\subsubsection{Equations and Inequalities}
In an equation, one side must of the equation must always be equal to the other side. So what you do to one side, you must do to both sides. The only exception to this rule is if you are doing something to one side that doesn't change the overall value, such as adding 0 or multiplying/dividing by 1.\\
Equations are often used to solve for an unknown.\\
Ex: $3x-6=7x+2$
\begin{flalign*}
& 3x-6-2=7x+2-2 &\\
& 3x-8=7x &\\
& 3x-8-3x=7x-3x &\\
& -8=4x &\\
& \frac{-8}{4}=\frac{4x}{4} &\\
& -2=x &
\end{flalign*}
Inequalities are stating when one side is larger than the other side.\\
Ex: $3x>6\Rightarrow x>2$\\
Note that when multiplying both sides by a negative, the sign switches directions.\\
Ex: $-x>4\Rightarrow x<-4$

\subsubsection{Like Terms}
A term is a group that is linked together through multiplication or division operations. For example, $3x$ would be a term. Terms are broken up by addition and subtraction operations. Like terms are when two terms have the same combination of variables such as $3a$ and $2a$. These can be added or subtracted from each other as they both have the variable $a$ in common. In the case of $3a$ and $4b$, they are not like terms and cannot be combined. Also note that $x$ and $x^2$ are not like terms, as the one consists of a singular $x$ where as the other one consists of two.\\
Ex: $6a+4b+3a-2ab+b=9a+5b-2ab$

\subsubsection{Distributive Property}
This is when you multiply a number or variable on the outside of a pair of parentheses to each term on the inside.\\
Ex: $3(x+y)=3x+3y$\\
In general,
$$a(b+c)=ab+bc$$
The product of binomials is similar to the distributive property, when you multiply two or more groups in parentheses.
$$(a+b)(c+d)=ac+ad+bc+bd$$
Ex: $(x+1)(x-2)=x^2+x-2x-2=x^2-x-2$

\subsubsection{Factoring}
Factoring is the opposite of the distributive property. It's where you take out a common factor from a set of numbers and put them in brackets.\\
Ex: $3x+6=3(x+2)$\\
The first step is to factor out a GCF if possible. Sometimes this is all that can be done, as with binomials.\\
Trinomials can prove tricky.\\
Ex: $x^2+5x+6$\\
You need to find two numbers that add to get the middle term and multiply to get the constant (3rd term).\\
To do this, we can find the factor pairs of the constant.\\
So for this example, they are $\{1,6\},\,\{-1,-6\},\,\{2,3\},\,\{-2,-3\}$\\
We can then deduce that the pair $\{2,3\}$ will work, so the answer will be\\ ${x^2+5x+6=(x+2)(x+3)}$\\
If the coefficient of the first term is greater than 1, we instead find two numbers that multiply to get the constant times said coefficient and add to get the middle term.\\
Ex: $2x^2+x-3$\\
We find factor pairs of $-6$ which are $\{-1,6\},\,\{1,-6\},\,\{2,-3\},\,\{-2,3\}$\\
The pair is $\{-2,3\}$ so our answer is\\
$2x^2+x-3=(2x+3)(x-1)$
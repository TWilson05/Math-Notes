\subsection{Introduction to Proofs}
\subsubsection{A First Proof}
The general method of writing a proof is to first write out some scratch work to solve the problem and then write a neat and structured proof which the reader will see.\\
Ex: Let $n$ be an integer. If $n$ is even then $n^2$ is even.\\
Scratch work:\\
We want to show that this implication is always true ($P\Ra Q)$\\
We assume the hypothesis, $n$ is even, is true (the false case doesn't tell us anything)\\
By the definition of even, $n=2k,\ k\in\Z$\\
$n^2=(2k)^2=4k^2=2(2k^2)$\\
Since $k\in\Z$ we know by an axiom that $2k^2\in\Z$ so, by the definition of even we know that $n^2$ is even.\\
We can rewrite this using modus ponens as
\begin{itemize}
    \item $n$ is even $\Ra$ $n=2k$ for some integer $k$
    \item $n=2k$ for some integer $k$ $\Ra$ $n^2=4k^2$
    \item $n=4k^2$ $\Ra$ $n^2$ is two times an integer
    \item $n^2$ is two times and integer $\Ra$ $n^2$ is even
\end{itemize}
Then the final proof would be written as follows:\\

Proof:
Let $n$ be an integer. If $n$ is even then $n^2$ is even\\
Assume that $n$ is an even number.\\
Hence we know that $n=2k$ for some $k\in\Z$\\
If follows that $n^2=4k^2=2(2k^2)$\\
Since $2k^2$ is an integer, it follows that $n^2$ is even $\square$

Ex2: Let $a,b,c\in\Z$. If $a|b$ and $b|c$ then $a|c$\\
Proof:\\
By definition of divisibility, $b=ka$ and $c=lb$ for $k,l\in\Z$.\\
We want to show $a|c$. That is $c=na$ for $n\in\Z$\\
Since $c=lb$ and $b=ka$, we know $c=lka$\\
Since $k,l\in\Z$ we know $kl\in\Z$ so we are done $\square$\\

Let $a,b,c\in\Z$. If $a|b$ and $b|c$ then $a|c$\\
Proof:\\
We start by assuming the hypothesis to be true\\
Assume that $a|b$ and $b|c$, so that $b=ka$ and $c=lb$ for some $k,l\in\Z$\\
It follows that $c=kla$\\
Since $kl\in\Z$, we know that $a|c$ as required $\square$
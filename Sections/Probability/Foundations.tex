\subsection{Foundations}

\subsubsection{Notation}
We can define the following notation for probability:
\begin{itemize}
    \item We denoted the \textit{state space} or \textit{universal set} as $\Omega$ or $S$.
    \item An \textit{event} is a ``nice" subset of the state space, i.e. an event $A$ fulfils $A\subseteq\Omega$. The set of all events is often denoted by $\mathcal{A},\ \mathcal{E},$ or $\mathcal{F}$.
    \item A \textit{probability measure} $\prob$ is a map taking events as argument with
    \begin{itemize}
        \item $\prob:\ \mathcal{A}\to[0,1]$
        \item $\prob(\Omega)=1$
        \item For a countable index set $I$ with $(A_i)_{i\in I}$ being disjoint events, we have $\prob\brround{\bigsqcup\limits_{i\in I}A_i}=\sum\limits_{i\in I}\prob(A_i)$
    \end{itemize}
\end{itemize}
Some more notation:\\
We use the square cups $\sqcup$ to denote the disjoint union of sets. This means that $A_1\sqcup A_2$ implies that the two sets have no overlap and they are disjoint: $A_1\cap A_2=\emptyset$.\\
Discrete Uniform Distribution:\\
We require that $\Omega$ is non empty and finite. If every event $\omega$ is equally likely, we have a discrete uniform distribution. This means that
\[\prob(\{\omega\})=\frac{1}{|\Omega|}\ \forall\omega\in\Omega\]
Then the probability of an event $A$ is
\[\prob(A)=\frac{|A|}{|\Omega|}\]
We call this setting a discrete uniform distribution or uniformly random.\\
Ex: For a die, we have $\Omega=\{1,2,3,4,5,6\}$ and $\prob(\{\omega\})=\frac{1}{6}$.\\
Ex2: For two die we have $\Omega=\brcurly{(1,1),(1,2),\ldots,(6,5),(6,6)}$ and the probability of rolling a 1, 2, or 3 is
\begin{align*}
    &\prob\brround{\brcurly{1,2,3}}=\prob\brround{\{1\}\sqcup\{2\}\sqcup\{3\}}=\prob(\{1\})+\prob(\{2\})+\prob(\{3\})=\frac{1}{6}+\frac{1}{6}+\frac{1}{6}=\frac{1}{2}
\end{align*}